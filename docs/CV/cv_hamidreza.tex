%%%%%%%%%%%%%%%%%%%%%%%%%%%%%%%%%%%%%%%%%
% Medium Length Professional CV
% LaTeX Template
% Version 2.0 (8/5/13)
%
% This template has been downloaded from:
% http://www.LaTeXTemplates.com
%
% Original author:
% Trey Hunner (http://www.treyhunner.com/)
%
% Important note:
% This template requires the resume.cls file to be in the same directory as the
% .tex file. The resume.cls file provides the resume style used for structuring the
% document.
%
%%%%%%%%%%%%%%%%%%%%%%%%%%%%%%%%%%%%%%%%%

%----------------------------------------------------------------------------------------
%	PACKAGES AND OTHER DOCUMENT CONFIGURATIONS
%----------------------------------------------------------------------------------------

\documentclass{resume} % Use the custom resume.cls style

\usepackage{hyperref}
\usepackage{graphicx}
\usepackage{fontawesome5}
\usepackage[usenames,dvipsnames]{xcolor}
\usepackage[left=0.5in,top=0.6in,right=0.5in,bottom=0.6in]{geometry} % Document margins
%\newcommand{\tab}[1]{\hspace{.2667\textwidth}\rlap{#1}}
%\newcommand{\itab}[1]{\hspace{0em}\rlap{#1}}

\address{+1 (437) 986-8970}
\address{\url{https://hamidrezakmk.github.io/}} % Your Personal webpage
\address{700 University Avenue} % Your secondary addess (optional)
\address{ hamidk@cs.toronto.edu }
\name{\Huge \textcolor{RoyalPurple!100}{\bf HAMID} \textcolor{RoyalPurple!100}{\scshape \bf KAMKARI}} % Your name


\begin{document}
% \begin{minipage}{0.48\textwidth}
% {\Huge \textcolor{RoyalPurple!100}{\bf Hamidreza} \textcolor{RoyalPurple!100}{\scshape \bf Kamkari}}\\
% \end{minipage} \begin{minipage}{0.40\textwidth}
% ‌\hspace{6mm} \faAt \ \ \textcolor{Black!70}{\href{mailto:hamidk@cs.toronto.edu}{hamidk@cs.toronto.edu}}\\
% ‌\hspace{6mm} \faGlobe \ \ \textcolor{Black!70}{\href{https://hamidrezakmk.github.io/}{\underline{HamidrezaKmK.github.io}}}\\
% ‌\hspace{6mm} \faPhone \ \ \textcolor{Black!70}{+1 (437) 986-8970}\\
% \end{minipage}
%\\
%----------------------------------------------------------------------------------------
%	EDUCATION SECTION
%----------------------------------------------------------------------------------------

\begin{rSection}{Education}

{\bf University of Toronto} \hfill \textcolor{Black!70}{\em 2022 - 2024} 
\\ \textcolor{Black!70}{Master of Science in Applied Computing (MScAC)}
\\ {\it Department of Computer Science}
\\
\begin{footnotesize}
{\it {\textcolor{Black!70}{\bf Ongoing Courses:}} (CSC2541)} Topics in Machine Learning: Introduction to Causality (A+), {\it (CSC2240)} Graphs, Matrices, and Continuous Optimization, {\it (CSC2701)} Communication for Computer Scientists,  {\it (CSC2541)} Advanced Topics in ML: Causal-aware Representation Learning, {\it (CSC2130)} Empirical Research Methods in Software Engineering
\end{footnotesize}
\\
\\
{\bf Sharif University of Technology} \hfill \textcolor{Black!70}{\em 2018 - 2022} 
\\ \textcolor{Black!70}{Bachelor of Science} \hfill \textcolor{Black!70}{ Overall GPA 19.22/20}
\\ {\it Department of Computer Engineering} \hfill \textcolor{Black!70}{Ranked Among the Top 10}
\\
\begin{footnotesize}
{\textcolor{Black!70}{\bf Courses:}} {\it (CE695)} Stochastic Processes, {\it (CE417)} Artificial Intelligence, {\it (CE494)} Introduction to Computational Biology, {\it (CE282)} Linear Algebra, {\it (CE181)} Fundamentals of Probability and Statistics, {\it (CE354)} Algorithm Design, {\it (CE415)} Theory of Formal Languages and Automata, {\it (MAT034)} Differential Equations
\end{footnotesize}
% \\{\bf AE Highschool} \hfill \textcolor{Black!70}{\em 2016 - 2018} 
% \\ Higher Secondary Education \hfill \textcolor{Black!70}{ Overall Score 19.23/20}
% \\ Mathematics, Physics
% \\
% \\{\bf Qazvin's School for Exceptional Talents (\href{https://en.wikipedia.org/wiki/National_Organization_for_Development_of_Exceptional_Talents}{\underline{NODET}})} \hfill \textcolor{Black!70}{\em2011 - 2016}
% \\ Secondary Education \hfill \textcolor{Black!70}{Overall Score 20/20}
% \\ State Board \hfill \textcolor{Black!70}{ State Rank-1}



\end{rSection}
%----------------------------------------------------------------------------------------
%	WORK EXPERIENCE SECTION
%----------------------------------------------------------------------------------------

\begin{rSection}{Research and Publication}

\begin{rSubsection}{Vector Institute}{\textcolor{Black!70}{\bf Toronto, Canada}}
{Causal Discovery based on Normalizing Flows}{\textcolor{Black!70}{November 2022 - Ongoing}}
\item {\tt (Preparing manuscript)}
\begin{small}
\item Used foundations in graph theory to come up with a new method of learning structures on directed acyclic graphs.
Implemented a normalizing flow-based causal discovery and inference structure under the supervision of Rahul Krishnan, used the
\item Pytorch Lightning framework for reproducibility, and within 3 months came up with an ICML2023 submission.
\end{small}
\end{rSubsection}

\begin{rSubsection}{Sharif University of Technology}{\textcolor{Black!70}{\bf Tehran, Iran}}
{Predicting Drug Combination Effects by Utilizing Multi-Omics Data}{\textcolor{Black!70}{January 2022 - September 2022}}
\begin{small}
\item Used Graph Neural Networks and Attention mechanisms to create a general state-of-the-art framework, named DeepDDR, for predicting drug dose response using SMILES representation of drugs.
\end{small}
\end{rSubsection}

%------------------------------------------------
\begin{rSubsection}{Maxplanck Institute of Informatics (MPI-INF)}{\textcolor{Black!70}{\bf Saarbrücken, Germany}}
{Convex Optimization - Algorithmic Perspective of Training Neural Networks}
{\textcolor{Black!70}{August 2020 - February 2022}}
\begin{small}
\item Using foundations in linear algebra, curated ideas that led to a creative dynamic system for a set of convex optimization problems. This led to a paper under review (STACS) on novel algorithms for tackling Semi-Definite programs using nature-inspired dynamics.
\item Undertook a second internship on the fundamentals of fine-grained bounds for fine-tuning simple overparameterized perceptrons. Using reduction from core algorithmic problems, proved fine-tuning problem on perceptrons is exponential w.r.t the dimension of the hidden units.
\end{small}
\end{rSubsection}

\begin{rSubsection}{Aalto University}{\textcolor{Black!70}{\bf Espoo, Finland}}
{RNA sequence design using Graph Neural Networks}
{\textcolor{Black!70}{July 2021 - September 2021}}{}
\begin{small}
\item Created a data-driven method to find RNA sequences that can fold into a certain secondary structure.
\item Modeled the problem as a Markov decision process and used Monte-Carlo Tree Search to speed up the search.
\item Helped improve the rollout phase using GNNs. Inspired by notions of connectivity in graph theory, obtained a suitable GNN with smart skip connections that autoregressively assigns molecules to nodes and obtains state-of-the-art results on a class of RNA structures.
\end{small}
\end{rSubsection}

\end{rSection}


%	EXAMPLE SECTION
%----------------------------------------------------------------------------------------

%----------------------------------------------------------------------------------------
%	TECHNICAL STRENGTHS SECTION
%----------------------------------------------------------------------------------------

\begin{rSection}{Work Experience}
\begin{small}
\begin{rSubsection}{Fanap IT Company}{\textcolor{Black!70}{\it January 2022 - August 2022}}{}{}
\itemsep -2pt
\item Research and development on deep learning methods to help restore poorly taken photos of dental panoramic images that prevents reshooting and additional x-ray exposure, and additionally, help with the tooth disease detection pipeline of dentists. 
\item Implemented a novel U-Net for dynamic range unification using Pytorch that can help panoramic image restoration.
\item Detectron2 Mask-RCNN for instance segmentation of teeth and treatments that can help computer-aided disease detection.
Created Demo using Docker and FastAPI for proof of concept and sold MVP to a client with three active radiology clinics in Tehran; all in approximately three months.

\end{rSubsection}
\begin{rSubsection}{National Olympiad in Informatics Committee}{\textcolor{Black!70}{\it September 2020 - December 2021}}{}{}
\itemsep -2pt
\item Curated and organized nation-wide competitive contests for talented students all across Iran; helped with the technical infrastructure of the online code judging system using CMS online judge.
\end{rSubsection}
\end{small}
\end{rSection}


\begin{rSection}{Teaching}

\begin{rSubsection}{Academic Teaching Assistance}{\textcolor{Black!70}{\bf University of Toronto - Sharif University of technology}}{}{}
\begin{small}
\item Introduction to Artificial Intelligence (CSC236) \href{https://www.cs.toronto.edu/~axgao/}{\underline{Alice Gao}} \hfill \textcolor{Black!70}{\it January 2023 - Ongoing}
\item Introduction to the theory of Computation (CSC236) \href{https://www.cs.toronto.edu/~fpitt/}{\underline{François Pitt}} \hfill \textcolor{Black!70}{\it September 2022 - December 2022}
\item Artificial Intelligence course (CE40417) \href{http://blogs.bu.edu/mhrohban/}{\underline{Mohammad Hossein Rohban}} \hfill \textcolor{Black!70}{\it September 2021 - January 2022}
\item {\bf Head}  of Data Structure and Algorithms course (CE40254) - \href{http://sharif.edu/~ghodsi/}{\underline{Mohammad Ghodsi}} \hfill \textcolor{Black!70}{\it January 2021 - June 2021}
\item Artificial Intelligence course (CE40417) \href{http://blogs.bu.edu/mhrohban/}{\underline{Mohammad Hossein Rohban}} \hfill \textcolor{Black!70}{\it January 2021 - June 2021}
\item Probability and Statistics course (CE40181) \href{https://scholar.google.com/citations?user=GbJMZLIAAAAJ&hl=en}{\underline{Ali Sharifi-Zarchi}} \hfill \textcolor{Black!70}{\it September 2020 - January 2021}
\item Discrete Structures course (CE40115) \href{https://scholar.google.com/citations?user=xuNJ-d8AAAAJ&hl=en}{\underline{Mohammad Ali Abam}} \hfill \textcolor{Black!70}{\it January 2020 - June 2020}
\item Advanced Algorithm design course (CE40354) \href{https://scholar.google.com/citations?user=GbJMZLIAAAAJ&hl=en}{\underline{Ali Sharifi-Zarchi}} \hfill \textcolor{Black!70}{\it January 2020 - June 2020} 
\item Data structure and Algorithms course (CE40254)
\end{small}
\end{rSubsection}
\begin{rSubsection}{Mentor-ship}{}{}{}
\itemsep -1pt
\begin{small}
\item \parbox{15cm}{Worked as Computer Olympiad Teacher in well-known Iranian high schools as well as a mentor at International Olympiad in Informatics (IOI) preparation camp for \href{https://ioi2019.az/}{International Olympiad in Informatics held in Baku, Azarbaijan}.}
\end{small}
\hfill 
\parbox{3cm}{\begin{flushright}
\begin{center}
\textcolor{Black!70}{\it January 2019 \\ February 2019}
\end{center}
\end{flushright}}
\end{rSubsection}
\end{rSection}

\begin{rSection}{Honors and Awards}
\begin{footnotesize}
\begin{minipage}[t]{1.6in}
\begin{center}
\vspace{0.2cm}
    \faLink \ \href{http://icpc.sharif.edu/2018/scoreboard/}{\bf \underline{ACM-ICPC}}\\
    Regional gold medal\\
    team ranked 3rd\\
    \textcolor{Black!70}{\it December 2018}
\end{center}
\end{minipage}
\hspace{0.1cm}
\begin{minipage}[t]{1.6in}
\begin{center}
\vspace{0.2cm}
\faLink \ \href{https://apio2018.ru/results/official-contest/}{\bf \underline{APIO}}\\
Asia-Pacific Informatics\\
bronze medal \\
\textcolor{Black!70}{\it May 2018}
\end{center}
\end{minipage}
\hspace{0.1cm}
\begin{minipage}[t]{1.6in}
\begin{center}
\vspace{0.2cm}
\faLink \ \href{https://www.info1cup.com/archive/2018/International_Round_Ranking.pdf}{\bf \underline{INFO-Cup}} \\
World-wide contests\\
gold medal\\
\textcolor{Black!70}{\it March 2018}
\end{center}
\end{minipage}
\hspace{0.1cm}
\begin{minipage}[t]{1.9in}
\begin{center}
\vspace{0.2cm}
\faLink \ {\bf \href{https://ioinformatics.org/journal/v11si_2017_25_33.pdf}{\underline{Computer Olympiad}}} \\
Ranked 5th in\\
national contests\\
\textcolor{Black!70}{\it September of 2016 \& 2017}
\end{center}
\end{minipage}
\end{footnotesize}
\end{rSection}
%----------------------------------------------------------------------------------------

\begin{rSection}{SKILLS}
\begin{minipage}[t]{3.2in}
\begin{center}
{\bf Programming Skills:} 

\begin{footnotesize} Python, Pytorch, Sklearn, Docker, C++ (for competetive programming), Java, MATLAB, \LaTeX. \end{footnotesize}
\end{center}
\end{minipage}
\hspace{0cm}
\begin{minipage}[t]{4in}
\begin{center}
{\bf Languages:}\\
\vspace{0.2cm}
\begin{footnotesize} Persian (Native) - English (Fluent) TOEFL iBT 116/120\\ Speaking: 28/30 - Writing: 29/30 - Reading: 30/30 - Listening: 29/30 
\end{footnotesize}
\end{center}
\end{minipage}
\end{rSection}

%\begin{rSection}{Miscellaneous} 
% I love traveling, watching movies and TV series. I also appreciate music and into composing in genres of rock and metal. 
 % \item Member, Athletics Team, IIT Kanpur. Attended Summer Sport Camp as a long jumper.
%\item Trained and disciplined in National Cadet Corps (NCC), IIT Kanpur for a year.
 %\item  Participated in Vijyoshi Camp 2012 organized at Indian Institute of Science, Bangalore.
 %\item Won 2nd position in Kho-Kho in Intramurals conducted by Physical Education Section, IIT Kanpur.
 %\item Pursued French as a second language during secondary school from Grade 6 to Grade 10. Also participated in French Song Competition and French G.K. Quiz in Class 10th. %

%\end{rSection}

\end{document}
